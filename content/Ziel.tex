\chapter{Zielsetzung}

\section{Ausgangslage}
Hintergrundinformationen zum Projekt.\\
Wie ist das Projekt zustande gekommen?\\
Was ist die Motivation für das Projekt?\\
Welcher Gewinn kann erzielt werden, wenn das Vorhaben gelöst ist?\\
Wurden Vorarbeiten durchgeführt? Welche? Was haben diese gezeigt?

\section{Auftrag}
Globaler Auftrag für das komplette Projekt als Fliesstext. In diesem Abschnitt geht es darum, zu zeigen, dass der Auftrag als ganzes erfasst wurde.

\section{Projektziele}
Spezifische Ziele für das aktuelle Semester. Meistens kann der globale Auftrag in Etappenziele unterteilt werden, welche für das aktuelle Semester definiert werden.\\

Es soll ersichtlich werden, wie die Ziele (ca. 3-4) helfen den globalen Auftrag zu erreichen.\\
Die Ziele sollen gemäss SMART Konzept geschrieben werden.\\
Wichtig, ein Ziel ist kein Arbeitspaket. Das heisst, ein Ziel ist ein Ergebnis und keine Tätigkeit.\\

Eine Anforderungsliste kann, muss aber nicht die Projektziele ergänzen. In der Anforderungsliste werden Pflicht- und Wunschziele definiert. Ferner kann es auch hilfreich sein, Nichtziele zu definieren, um klar abzugrenzen, was im Projekt explizit nicht erledigt wird.


\section{Systemgrenze}
Die Systemgrenze ist optional.\\
In der Systemgrenze wird die Schnittstelle zwischen der erarbeiteten Lösung und der Umwelt definiert.\\
Bei einer Software-Lösung kann es ein Klassendiagramm sein.\\
Bei einem Hardware-Projekt kann es als Blockdiagramm realisiert werden.


\section{Vorgehen}
Das Vorgehen ist optional. Hier kann darauf aufmerksam gemacht werden, dass gewisse Projektziele aufeinander aufbauen und dabei Risiken entstehen.






