\chapter{Projektplanung}

\section{Zeitbudget}
Wie viel Stunden stehen für das Projekt zur Verfügung?\\

Semester Beginn.\\
Semester Ende.

\section{Zeitplan}
    \begin{ganttchart}
        [
        hgrid,
        vgrid={*6{white},*1{dotted}},
        x unit=0.08cm,
        y unit chart=0.8cm,
            time slot format=isodate,
        title label font=\footnotesize,
        bar label font=\footnotesize,
        bar/.append style={fill=blue!30},
        milestone label font=\footnotesize,
        milestone/.append style={fill=red},
        milestone/.append style={xscale=5} 
        ]{2024-01-01}{2024-06-20} % Start und Ende
        \label{gantt:01}
        % Jahre
        \gantttitlecalendar{year, month=shortname}\\
        % Arbeitspakete und Meilensteine
        \ganttbar{Arbeitspaket 1}{2024-01-01}{2024-02-15}\\
        \ganttmilestone{Meilenstein 1}{2024-01-15}\\
        \ganttbar{Arbeitspaket 2}{2024-03-01}{2024-05-15}\\        
        \ganttmilestone{Meilenstein 2}{2024-05-15}
    \end{ganttchart}


\subsection{Meilensteine}

    \begin{tabular*}{\textwidth}{c l p{4cm} l}
        \toprule
        Nr. & Meilenstein & & Abgabedatum \\
        \midrule \midrule
        \ref{Ms:Bsp01}  & MS A   && 01.15.2024\\
        \ref{Ms:Bsp02}  & MS B   && 05.15.2024\\
        \bottomrule
    \end{tabular*}
    \\ [1em]
    
    \Meilenstein{MS A}\label{Ms:Bsp01}
        Beschrieb\\
        Datum: 15.01.2024\\
    
    \Meilenstein{MS B}\label{Ms:Bsp02}
        Beschrieb\\
        Datum: 15.05.2024\\


\subsection{Detaillierte Arbeits- und Zeitplanung}

    \Arbeitspaket{Arbeitspaket 1}
        Inhalt, Ergebins, Meilenstein\\
        
    \Arbeitspaket{Arbeitspaket 2}
        Inhalt, Ergebins, Meilenstein\\














